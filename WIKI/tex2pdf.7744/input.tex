\documentclass[]{article}
\usepackage{lmodern}
\usepackage{amssymb,amsmath}
\usepackage{ifxetex,ifluatex}
\usepackage{fixltx2e} % provides \textsubscript
\ifnum 0\ifxetex 1\fi\ifluatex 1\fi=0 % if pdftex
  \usepackage[T1]{fontenc}
  \usepackage[utf8]{inputenc}
\else % if luatex or xelatex
  \ifxetex
    \usepackage{mathspec}
  \else
    \usepackage{fontspec}
  \fi
  \defaultfontfeatures{Ligatures=TeX,Scale=MatchLowercase}
\fi
% use upquote if available, for straight quotes in verbatim environments
\IfFileExists{upquote.sty}{\usepackage{upquote}}{}
% use microtype if available
\IfFileExists{microtype.sty}{%
\usepackage{microtype}
\UseMicrotypeSet[protrusion]{basicmath} % disable protrusion for tt fonts
}{}
\usepackage[unicode=true]{hyperref}
\hypersetup{
            pdfborder={0 0 0},
            breaklinks=true}
\urlstyle{same}  % don't use monospace font for urls
\usepackage{longtable,booktabs}
\IfFileExists{parskip.sty}{%
\usepackage{parskip}
}{% else
\setlength{\parindent}{0pt}
\setlength{\parskip}{6pt plus 2pt minus 1pt}
}
\setlength{\emergencystretch}{3em}  % prevent overfull lines
\providecommand{\tightlist}{%
  \setlength{\itemsep}{0pt}\setlength{\parskip}{0pt}}
\setcounter{secnumdepth}{0}
% Redefines (sub)paragraphs to behave more like sections
\ifx\paragraph\undefined\else
\let\oldparagraph\paragraph
\renewcommand{\paragraph}[1]{\oldparagraph{#1}\mbox{}}
\fi
\ifx\subparagraph\undefined\else
\let\oldsubparagraph\subparagraph
\renewcommand{\subparagraph}[1]{\oldsubparagraph{#1}\mbox{}}
\fi

\date{}

\begin{document}

\section{ChatScript System Variables and Engine-defined
Concepts}\label{chatscript-system-variables-and-engine-defined-concepts}

\begin{quote}
© Bruce Wilcox, gowilcox@gmail.com brilligunderstanding.com
\end{quote}

\begin{quote}
Revision 9/25/2016 cs6.84
\end{quote}

\begin{itemize}
\tightlist
\item
  \href{ChatScript-System-Variables-and-Engine-defined-Concepts.md\#engine-defined-concepts}{Engine-defined
  Concepts}
\item
  \href{ChatScript-System-Variables-and-Engine-defined-Concepts.md\#system-variables}{System
  Variables}
\item
  \href{ChatScript-System-Variables-and-Engine-defined-Concepts.md\#control-over-input}{Control
  over Input}
\item
  \href{ChatScript-System-Variables-and-Engine-defined-Concepts.md\#interchange-variables}{Interchange
  Variables}
\end{itemize}

\section{Engine-defined concepts}\label{engine-defined-concepts}

In addition to concepts defined in script files, the system
automatically defines a bunch of dictionary-based sets as well as
dynamically computed concept members.

\begin{longtable}[]{@{}ll@{}}
\toprule
\begin{minipage}[b]{0.22\columnwidth}\raggedright\strut
set\strut
\end{minipage} & \begin{minipage}[b]{0.10\columnwidth}\raggedright\strut
description\strut
\end{minipage}\tabularnewline
\midrule
\endhead
\begin{minipage}[t]{0.22\columnwidth}\raggedright\strut
\texttt{\textasciitilde{}web\_url}\strut
\end{minipage} & \begin{minipage}[t]{0.10\columnwidth}\raggedright\strut
word is a web url\strut
\end{minipage}\tabularnewline
\begin{minipage}[t]{0.22\columnwidth}\raggedright\strut
\texttt{\textasciitilde{}email\_url}\strut
\end{minipage} & \begin{minipage}[t]{0.10\columnwidth}\raggedright\strut
word is an email address\strut
\end{minipage}\tabularnewline
\begin{minipage}[t]{0.22\columnwidth}\raggedright\strut
\texttt{\textasciitilde{}kindergarten}\strut
\end{minipage} & \begin{minipage}[t]{0.10\columnwidth}\raggedright\strut
word learned early in life\strut
\end{minipage}\tabularnewline
\begin{minipage}[t]{0.22\columnwidth}\raggedright\strut
\texttt{\textasciitilde{}grade1\_2}\strut
\end{minipage} & \begin{minipage}[t]{0.10\columnwidth}\raggedright\strut
word learned in these grades\strut
\end{minipage}\tabularnewline
\begin{minipage}[t]{0.22\columnwidth}\raggedright\strut
\texttt{\textasciitilde{}grade3\_4}\strut
\end{minipage} & \begin{minipage}[t]{0.10\columnwidth}\raggedright\strut
word learned in these grades\strut
\end{minipage}\tabularnewline
\begin{minipage}[t]{0.22\columnwidth}\raggedright\strut
\texttt{\textasciitilde{}grade\_5-6}\strut
\end{minipage} & \begin{minipage}[t]{0.10\columnwidth}\raggedright\strut
word learned in these grades. Unmarked words are learned even
later\strut
\end{minipage}\tabularnewline
\begin{minipage}[t]{0.22\columnwidth}\raggedright\strut
\texttt{\textasciitilde{}utf8}\strut
\end{minipage} & \begin{minipage}[t]{0.10\columnwidth}\raggedright\strut
word has nonascii characters\strut
\end{minipage}\tabularnewline
\begin{minipage}[t]{0.22\columnwidth}\raggedright\strut
\texttt{\textasciitilde{}daynumber}\strut
\end{minipage} & \begin{minipage}[t]{0.10\columnwidth}\raggedright\strut
word could be a number of a day in a month\strut
\end{minipage}\tabularnewline
\begin{minipage}[t]{0.22\columnwidth}\raggedright\strut
\texttt{\textasciitilde{}yearnumber}\strut
\end{minipage} & \begin{minipage}[t]{0.10\columnwidth}\raggedright\strut
word could be the number of a recent year\strut
\end{minipage}\tabularnewline
\begin{minipage}[t]{0.22\columnwidth}\raggedright\strut
\texttt{\textasciitilde{}dateinfo}\strut
\end{minipage} & \begin{minipage}[t]{0.10\columnwidth}\raggedright\strut
phrase is month day year of some kind\strut
\end{minipage}\tabularnewline
\begin{minipage}[t]{0.22\columnwidth}\raggedright\strut
\texttt{\textasciitilde{}kelvin}\strut
\end{minipage} & \begin{minipage}[t]{0.10\columnwidth}\raggedright\strut
temperature marker\strut
\end{minipage}\tabularnewline
\begin{minipage}[t]{0.22\columnwidth}\raggedright\strut
\texttt{\textasciitilde{}celcius}\strut
\end{minipage} & \begin{minipage}[t]{0.10\columnwidth}\raggedright\strut
temperature marker\strut
\end{minipage}\tabularnewline
\begin{minipage}[t]{0.22\columnwidth}\raggedright\strut
\texttt{\textasciitilde{}fahrenheit}\strut
\end{minipage} & \begin{minipage}[t]{0.10\columnwidth}\raggedright\strut
temperature marker\strut
\end{minipage}\tabularnewline
\begin{minipage}[t]{0.22\columnwidth}\raggedright\strut
\texttt{\textasciitilde{}twitter\_name}\strut
\end{minipage} & \begin{minipage}[t]{0.10\columnwidth}\raggedright\strut
twitter user name\strut
\end{minipage}\tabularnewline
\begin{minipage}[t]{0.22\columnwidth}\raggedright\strut
\texttt{\textasciitilde{}hashtag\_label}\strut
\end{minipage} & \begin{minipage}[t]{0.10\columnwidth}\raggedright\strut
twitter topic reference\strut
\end{minipage}\tabularnewline
\bottomrule
\end{longtable}

\subsection{\texorpdfstring{Interjections, ``discourse acts'', and
concept
sets}{Interjections, discourse acts, and concept sets}}\label{interjections-discourse-acts-and-concept-sets}

Some words and phrases have interpretations based on whether they are at
sentence start or not. E.g., \emph{good day, mate} and \emph{It is a
good day} are different for \emph{good day}.

Likewise sure and I am sure are different. Words that have a different
meaning at the start of a sentence are commonly called interjections.

In ChatScript these are defined by the
\texttt{livedata/interjections.txt} file. In addition, the file augments
this concept with ``discourse acts'', phrases that are like an
interjection. All interjections and discourse acts map to concept sets,
which come thru as the user input instead of what they wrote. For
example yes and sure and of course are all treated as meaning the
discourse act of agreement in the interjections file. So you don't see
yes, I will go coming out of the engine.

The interjections file will remap that to the sentence
\textasciitilde{}yes, breaking off that into its own sentence, followed
by I will go as a new sentence.

These generic interjections (which are open to author control via
interjections.txt) are:
\texttt{\textasciitilde{}yes},\texttt{\textasciitilde{}no},\texttt{\textasciitilde{}emomaybe},\texttt{\textasciitilde{}emohello},\texttt{\textasciitilde{}emogoodbye},\texttt{\textasciitilde{}emohowzit},\texttt{\textasciitilde{}emothanks},
\texttt{\textasciitilde{}emolaugh},\texttt{\textasciitilde{}emohappy},\texttt{\textasciitilde{}emosad},\texttt{\textasciitilde{}emosurprise},
\texttt{\textasciitilde{}emomisunderstand},
\texttt{\textasciitilde{}emoskeptic},\texttt{\textasciitilde{}emoignorance},\texttt{\textasciitilde{}emobeg},
\texttt{\textasciitilde{}emobored},
\texttt{\textasciitilde{}emopain},\texttt{\textasciitilde{}emoangry},
\texttt{\textasciitilde{}emocurse},\texttt{\textasciitilde{}emodisgust},\texttt{\textasciitilde{}emoprotest},
\texttt{\textasciitilde{}emoapology},\texttt{\textasciitilde{}emomutual}

Because all interjections at the start of a sentence are broken off into
their own sentence, this kind of pattern does not work:

\begin{verbatim}
u: (~yes _*)
\end{verbatim}

You cannot capture the rest of the sentence here, because it will be
part of the next sentence instead. This means interjections act somewhat
differently from other concepts.

If you use a word in a pattern which may get remapped on input, the
script compiler will issue a warning. Likely you should use the remapped
name instead.

The following concepts are triggered by exactly repeating either the
chatbot or oneself (to a repeat count of how often repeated). Repeats
are within a recency window of about 20 volleys.
\texttt{\textasciitilde{}repeatme},
\texttt{\textasciitilde{}repeatinput1},
\texttt{\textasciitilde{}repeatinput2},
\texttt{\textasciitilde{}repeatinput3},
\texttt{\textasciitilde{}repeatinput4},
\texttt{\textasciitilde{}repeatinput5},
\texttt{\textasciitilde{}repeatinput6},

\subsection{POS (Part of Speech) Tags}\label{pos-part-of-speech-tags}

Words will have pos-tags attached, specififying both generic and
specific tag attributes, eg., \texttt{\textasciitilde{}noun} and
\texttt{\textasciitilde{}noun\_singular}.

\subsubsection{Genric Specifics}\label{genric-specifics}

\texttt{\textasciitilde{}noun},
\texttt{\textasciitilde{}noun\_singular},
\texttt{\textasciitilde{}noun\_plural},
\texttt{\textasciitilde{}noun\_proper\_singular},
\texttt{\textasciitilde{}noun\_proper\_plural},
\texttt{\textasciitilde{}noun\_gerund},
\texttt{\textasciitilde{}noun\_number},
\texttt{\textasciitilde{}noun\_infinitive},
\texttt{\textasciitilde{}noun\_omitted\_adjective},
\texttt{\textasciitilde{}verb}, \texttt{\textasciitilde{}verb\_present},
\texttt{\textasciitilde{}verb\_present\_3ps},
\texttt{\textasciitilde{}verb\_infinitive},
\texttt{\textasciitilde{}verb\_present\_participle},
\texttt{\textasciitilde{}verb\_past},
\texttt{\textasciitilde{}verb\_past\_participle},
\texttt{\textasciitilde{}aux\_verb},
\texttt{\textasciitilde{}aux\_verb\_present},
\texttt{\textasciitilde{}aux\_verb\_past},
\texttt{\textasciitilde{}aux\_verb\_future}
(\texttt{\textasciitilde{}aux\_verb\_tenses}),
\texttt{\textasciitilde{}aux\_be}, \texttt{\textasciitilde{}aux\_have},
\texttt{\textasciitilde{}aux\_do}

Auxilliary verbs are segmented into normal ones and special ones. Normal
ones give their tense directly. Special ones give their root word. The
tense of the be/have/do verbs can be had via \texttt{\^{}properties()}
and testing for verb tenses

\texttt{\textasciitilde{}adjective},
\texttt{\textasciitilde{}adjective\_normal},
\texttt{\textasciitilde{}adjective\_number},
\texttt{\textasciitilde{}adjective\_noun},
\texttt{\textasciitilde{}adjective\_participle}

Adjectives in comparative form will also have
\texttt{\textasciitilde{}more\_form} or
\texttt{\textasciitilde{}most\_form}. \texttt{\textasciitilde{}adverb},
\texttt{\textasciitilde{}adverb\_normal}

Adverbs in comparative form will also have
\texttt{\textasciitilde{}more\_form} or
\texttt{\textasciitilde{}most\_form}. \texttt{\textasciitilde{}pronoun},
\texttt{\textasciitilde{}pronoun\_subject},
\texttt{\textasciitilde{}pronoun\_object},
\texttt{\textasciitilde{}conjunction\_bits},
\texttt{\textasciitilde{}conjunction\_coordinate} ,
\texttt{\textasciitilde{}conjunction\_subordinate},
\texttt{\textasciitilde{}determiner\_bits},
\texttt{\textasciitilde{}determiner},
\texttt{\textasciitilde{}pronoun\_possessive},
\texttt{\textasciitilde{}predeterminer},
\texttt{\textasciitilde{}possessive} (covers `and 's at end of word),
\texttt{\textasciitilde{}to\_infinitive} (``to'' when used before a noun
infinitive), \texttt{\textasciitilde{}preposition},
\texttt{\textasciitilde{}particle} (free-floating preposition tied to
idiomatic verb), \texttt{\textasciitilde{}comma},
\texttt{\textasciitilde{}quote} (covers' and " when not embedded in a
word), \texttt{\textasciitilde{}paren} (covers opening and closing
parens), \texttt{\textasciitilde{}foreign\_word} (some unknown word),
\texttt{\textasciitilde{}there\_existential} (the word there used
existentially),

In addition to normal generic kinds of pos tags, words which are serving
a pos-tag role different from their putative word type are marked as
members of the major tag they act as part of. E.g,

\texttt{\textasciitilde{}noun\_gerund} -- verb used as a
\textasciitilde{}noun \texttt{\textasciitilde{}noun\_infinitive} -- verb
used as a \textasciitilde{}noun
\texttt{\textasciitilde{}noun\_omitted\_adjective} -- an adjective used
as a collective noun (eg the beautiful are kind)

\texttt{\textasciitilde{}adjectival\_noun} (noun used as adjective like
bank ``bank teller'') \texttt{\textasciitilde{}adjective\_participle}
(verb participle used as an adjective)

For \texttt{\textasciitilde{}noun\_gerund} in \emph{I like swimming} the
verb gerund \emph{swimming} is treated as a noun (hence called
noun-gerund) but retains verb sense when matching keywords tagged with
part-of-speech (i.e., it would match \texttt{swim\textasciitilde{}v} as
well as \texttt{swim\textasciitilde{}n}).

\texttt{\textasciitilde{}number} is not a part of speech, but is
comprise of \texttt{\textasciitilde{}noun\_number} (a normal number
value like \emph{17} or \emph{seventeen}) and
\texttt{\textasciitilde{}adjective\_number} (also a normal numeral value
and also \texttt{\textasciitilde{}placenumber}) like first.

To can be a preposition or it can be special. When used in the
infinitive phrase To go, it is marked
\texttt{\textasciitilde{}to\_infinitive} and is followed by
\texttt{\textasciitilde{}noun\_infinitive}.

\texttt{\textasciitilde{}verb\_infinitive} refers to a match on the
infinitive form of the verb (I hear John sing or I will sing).

\texttt{\textasciitilde{}There\_existential} refers to the use of where
not involving location, meaning the existence of, as in There is no
future.

\texttt{\textasciitilde{}Particle} refers to a preposition piece of a
compound verb idiom which allows being separated from the verb. If you
say \emph{I will call off the meeting}, call\_off is the composite verb
and is a single token. But if you split it as in \emph{I will call the
meeting off}, then there are two tokens. The original form of the verb
will be call and the canonical form of the verb will be call\_off, while
the free-standing off will be labeled
\texttt{\textasciitilde{}particle}.

\texttt{\textasciitilde{}verb\_present} will be used for normal present
verbs not in third person singular like I walk and
\texttt{\textasciitilde{}verb\_present\_3ps} will be used for things
like he walks

\texttt{\textasciitilde{}possesive} refers to `s and ` that indicate
possession, while possessive pronouns get their own labeling
\texttt{\textasciitilde{}pronoun\_possessive}.

\texttt{\textasciitilde{}pronoun\_subject} is a pronoun used as a
subject (like he) while pronoun\_object refers to objective form like
(him)

Individual words serve roles in the parse of a sentence, which are
retrievable. These include:

\texttt{\textasciitilde{}mainsubject},
\texttt{\textasciitilde{}mainverb},
\texttt{\textasciitilde{}mainindirect},
\texttt{\textasciitilde{}maindirect},
\texttt{\textasciitilde{}subject2}, \texttt{\textasciitilde{}verb2},
\texttt{\textasciitilde{}indirectobject2},
\texttt{\textasciitilde{}object2},
\texttt{\textasciitilde{}subject\_complement} -- (adjective object of
sentence involving linking verb),
\texttt{\textasciitilde{}object\_complement} -- (2ndary noun or
infinitive verb filling modifying mainobject or object2),
\texttt{\textasciitilde{}conjunct\_noun},
\texttt{\textasciitilde{}conjunct\_verb},
\texttt{\textasciitilde{}conjunct\_adjective},
\texttt{\textasciitilde{}conjunct\_adverb}
\texttt{\textasciitilde{}conjunct\_phrase},
\texttt{\textasciitilde{}conjunct\_clause},
\texttt{\textasciitilde{}conjunct\_sentence},
\texttt{\textasciitilde{}postnominalAdjective} - adjective occuring
AFTER the noun it modified, \texttt{\textasciitilde{}reflexive} -
(reflexive pronouns), \texttt{\textasciitilde{}not},
\texttt{\textasciitilde{}address} -- noun used as addressee of sentence,
\texttt{\textasciitilde{}appositive} -- noun restating and modifying
prior noun, \texttt{\textasciitilde{}absolutephrase} -- special phrase
describing whole sentence, \texttt{\textasciitilde{}omittedtimeprep} --
modified time word used as phrase but lacking preposition (Next tuesday
I will go), \texttt{\textasciitilde{}phrase} -- a prepositional phrase
start (except, \texttt{\textasciitilde{}clause} -- a subordinate clause
start, \texttt{\textasciitilde{}verbal} -- a verb phrase.

\section{System Variables}\label{system-variables}

The system has some predefined variables which you can generally test
and use but not normally assign to. These all begin with \texttt{\%} .
Ones that are reasonable to set are written in bold underline. Boolean
values are always \texttt{1} or \texttt{null} on returns. \texttt{1} or
\texttt{0} if you are setting them.

\subsection{Date \& Time \& Numbers}\label{date-time-numbers}

\begin{longtable}[]{@{}ll@{}}
\toprule
\begin{minipage}[b]{0.12\columnwidth}\raggedright\strut
variable\strut
\end{minipage} & \begin{minipage}[b]{0.61\columnwidth}\raggedright\strut
description\strut
\end{minipage}\tabularnewline
\midrule
\endhead
\begin{minipage}[t]{0.12\columnwidth}\raggedright\strut
\texttt{\%date}\strut
\end{minipage} & \begin{minipage}[t]{0.61\columnwidth}\raggedright\strut
one or two digit day of the month\strut
\end{minipage}\tabularnewline
\begin{minipage}[t]{0.12\columnwidth}\raggedright\strut
\texttt{\%day}\strut
\end{minipage} & \begin{minipage}[t]{0.61\columnwidth}\raggedright\strut
Sunday, etc\strut
\end{minipage}\tabularnewline
\begin{minipage}[t]{0.12\columnwidth}\raggedright\strut
\texttt{\%daynumber}\strut
\end{minipage} & \begin{minipage}[t]{0.61\columnwidth}\raggedright\strut
0-6 where 0 = Sunday\strut
\end{minipage}\tabularnewline
\begin{minipage}[t]{0.12\columnwidth}\raggedright\strut
\texttt{\%fulltime}\strut
\end{minipage} & \begin{minipage}[t]{0.61\columnwidth}\raggedright\strut
seconds representing the current time and date (Unix epoch time)\strut
\end{minipage}\tabularnewline
\begin{minipage}[t]{0.12\columnwidth}\raggedright\strut
\texttt{\%timenumbers}\strut
\end{minipage} & \begin{minipage}[t]{0.61\columnwidth}\raggedright\strut
completely consistent full time info in numbers that you can do
\texttt{\_0\ =\ \^{}burst(\%timenumbers)}to get \texttt{\_0} =seconds
(2digit) \texttt{\_1}=minutes (2digit) \texttt{\_2}=hours (2digit)
\texttt{\_3}=dayinweek(0-6 Sunday=0) \texttt{\_4}=dateinmonth (1-31)
\texttt{\_5}=month(0-11 January=0) \texttt{\_6}=year. You need to get it
simultaneously if you want to do accurate things with current time,
since retrieving \%hour \%minute separately allows time to change
between calls\strut
\end{minipage}\tabularnewline
\begin{minipage}[t]{0.12\columnwidth}\raggedright\strut
\texttt{\%leapyear}\strut
\end{minipage} & \begin{minipage}[t]{0.61\columnwidth}\raggedright\strut
boolean if current year is a leap year\strut
\end{minipage}\tabularnewline
\begin{minipage}[t]{0.12\columnwidth}\raggedright\strut
\texttt{\%daylightsavings}\strut
\end{minipage} & \begin{minipage}[t]{0.61\columnwidth}\raggedright\strut
boolean if current within daylight savings\strut
\end{minipage}\tabularnewline
\begin{minipage}[t]{0.12\columnwidth}\raggedright\strut
\texttt{\%minute}\strut
\end{minipage} & \begin{minipage}[t]{0.61\columnwidth}\raggedright\strut
0-59\strut
\end{minipage}\tabularnewline
\begin{minipage}[t]{0.12\columnwidth}\raggedright\strut
\texttt{\%month}\strut
\end{minipage} & \begin{minipage}[t]{0.61\columnwidth}\raggedright\strut
1-12 (January = 1)\strut
\end{minipage}\tabularnewline
\begin{minipage}[t]{0.12\columnwidth}\raggedright\strut
\texttt{\%monthname}\strut
\end{minipage} & \begin{minipage}[t]{0.61\columnwidth}\raggedright\strut
January, etc\strut
\end{minipage}\tabularnewline
\begin{minipage}[t]{0.12\columnwidth}\raggedright\strut
\texttt{\%second}\strut
\end{minipage} & \begin{minipage}[t]{0.61\columnwidth}\raggedright\strut
0-59\strut
\end{minipage}\tabularnewline
\begin{minipage}[t]{0.12\columnwidth}\raggedright\strut
\texttt{\%volleytime}\strut
\end{minipage} & \begin{minipage}[t]{0.61\columnwidth}\raggedright\strut
number of seconds of computation since volley input started\strut
\end{minipage}\tabularnewline
\begin{minipage}[t]{0.12\columnwidth}\raggedright\strut
\texttt{\%time}\strut
\end{minipage} & \begin{minipage}[t]{0.61\columnwidth}\raggedright\strut
hh:mm in military 24-hour time\strut
\end{minipage}\tabularnewline
\begin{minipage}[t]{0.12\columnwidth}\raggedright\strut
\texttt{\%week}\strut
\end{minipage} & \begin{minipage}[t]{0.61\columnwidth}\raggedright\strut
1-5 (week of the month)\strut
\end{minipage}\tabularnewline
\begin{minipage}[t]{0.12\columnwidth}\raggedright\strut
\texttt{\%year}\strut
\end{minipage} & \begin{minipage}[t]{0.61\columnwidth}\raggedright\strut
e.g., 2011\strut
\end{minipage}\tabularnewline
\begin{minipage}[t]{0.12\columnwidth}\raggedright\strut
\texttt{\%rand}\strut
\end{minipage} & \begin{minipage}[t]{0.61\columnwidth}\raggedright\strut
get a random number from 1 to 100 inclusive\strut
\end{minipage}\tabularnewline
\bottomrule
\end{longtable}

Time and date information are normally local, relative to the system
clock of the machine CS is running on. See \$cs\_utcoffset for adjusting
time based on relationship to utc (e.g your server is in Virginia and
you are in Colorado).

\subsection{User Input}\label{user-input}

\begin{longtable}[]{@{}ll@{}}
\toprule
\begin{minipage}[b]{0.12\columnwidth}\raggedright\strut
variable\strut
\end{minipage} & \begin{minipage}[b]{0.10\columnwidth}\raggedright\strut
description\strut
\end{minipage}\tabularnewline
\midrule
\endhead
\begin{minipage}[t]{0.12\columnwidth}\raggedright\strut
\texttt{\%bot}\strut
\end{minipage} & \begin{minipage}[t]{0.10\columnwidth}\raggedright\strut
current bot responding\strut
\end{minipage}\tabularnewline
\begin{minipage}[t]{0.12\columnwidth}\raggedright\strut
\texttt{\%revisedinput}\strut
\end{minipage} & \begin{minipage}[t]{0.10\columnwidth}\raggedright\strut
Boolean is current input from \^{}input not direct from user\strut
\end{minipage}\tabularnewline
\begin{minipage}[t]{0.12\columnwidth}\raggedright\strut
\texttt{\%command}\strut
\end{minipage} & \begin{minipage}[t]{0.10\columnwidth}\raggedright\strut
Boolean was the user input a command\strut
\end{minipage}\tabularnewline
\begin{minipage}[t]{0.12\columnwidth}\raggedright\strut
\texttt{\%foreign}\strut
\end{minipage} & \begin{minipage}[t]{0.10\columnwidth}\raggedright\strut
Boolean is bulk of the sentence composed of foreign words\strut
\end{minipage}\tabularnewline
\begin{minipage}[t]{0.12\columnwidth}\raggedright\strut
\texttt{\%impliedyou}\strut
\end{minipage} & \begin{minipage}[t]{0.10\columnwidth}\raggedright\strut
Boolean was the user input having you as implied subject\strut
\end{minipage}\tabularnewline
\begin{minipage}[t]{0.12\columnwidth}\raggedright\strut
\texttt{\%input}\strut
\end{minipage} & \begin{minipage}[t]{0.10\columnwidth}\raggedright\strut
the count of the number of volleys this user has made ever\strut
\end{minipage}\tabularnewline
\begin{minipage}[t]{0.12\columnwidth}\raggedright\strut
\texttt{\%ip}\strut
\end{minipage} & \begin{minipage}[t]{0.10\columnwidth}\raggedright\strut
ip address supplied\strut
\end{minipage}\tabularnewline
\begin{minipage}[t]{0.12\columnwidth}\raggedright\strut
\texttt{\%length}\strut
\end{minipage} & \begin{minipage}[t]{0.10\columnwidth}\raggedright\strut
the length in tokens of the current sentence\strut
\end{minipage}\tabularnewline
\begin{minipage}[t]{0.12\columnwidth}\raggedright\strut
\texttt{\%more}\strut
\end{minipage} & \begin{minipage}[t]{0.10\columnwidth}\raggedright\strut
Boolean is there another sentence after this\strut
\end{minipage}\tabularnewline
\begin{minipage}[t]{0.12\columnwidth}\raggedright\strut
\texttt{\%morequestion}\strut
\end{minipage} & \begin{minipage}[t]{0.10\columnwidth}\raggedright\strut
Boolean is there a ? or question word in the pending sentences\strut
\end{minipage}\tabularnewline
\begin{minipage}[t]{0.12\columnwidth}\raggedright\strut
\texttt{\%originalinput}\strut
\end{minipage} & \begin{minipage}[t]{0.10\columnwidth}\raggedright\strut
all sentences user passed into volley, before adjusted in any way except
OOB data is stripped off\strut
\end{minipage}\tabularnewline
\begin{minipage}[t]{0.12\columnwidth}\raggedright\strut
\texttt{\%originalsentence}\strut
\end{minipage} & \begin{minipage}[t]{0.10\columnwidth}\raggedright\strut
the current sentence after tokenization but before any adjustments\strut
\end{minipage}\tabularnewline
\begin{minipage}[t]{0.12\columnwidth}\raggedright\strut
\texttt{\%parsed}\strut
\end{minipage} & \begin{minipage}[t]{0.10\columnwidth}\raggedright\strut
Boolean was current input parsed successfully\strut
\end{minipage}\tabularnewline
\begin{minipage}[t]{0.12\columnwidth}\raggedright\strut
\texttt{\%question}\strut
\end{minipage} & \begin{minipage}[t]{0.10\columnwidth}\raggedright\strut
Boolean was the user input a question -- same as ? in a pattern\strut
\end{minipage}\tabularnewline
\begin{minipage}[t]{0.12\columnwidth}\raggedright\strut
\texttt{\%quotation}\strut
\end{minipage} & \begin{minipage}[t]{0.10\columnwidth}\raggedright\strut
Boolean is current input a quotation\strut
\end{minipage}\tabularnewline
\begin{minipage}[t]{0.12\columnwidth}\raggedright\strut
\texttt{\%sentence}\strut
\end{minipage} & \begin{minipage}[t]{0.10\columnwidth}\raggedright\strut
Boolean does it seem like a sentence (subject/verb or command)\strut
\end{minipage}\tabularnewline
\begin{minipage}[t]{0.12\columnwidth}\raggedright\strut
\texttt{\%tense}\strut
\end{minipage} & \begin{minipage}[t]{0.10\columnwidth}\raggedright\strut
past , present, or future simple tense (present perfect is a past
tense)\strut
\end{minipage}\tabularnewline
\begin{minipage}[t]{0.12\columnwidth}\raggedright\strut
\texttt{\%user}\strut
\end{minipage} & \begin{minipage}[t]{0.10\columnwidth}\raggedright\strut
user login name supplied\strut
\end{minipage}\tabularnewline
\begin{minipage}[t]{0.12\columnwidth}\raggedright\strut
\texttt{\%userfirstline}\strut
\end{minipage} & \begin{minipage}[t]{0.10\columnwidth}\raggedright\strut
value of \%input that is at the start of this conversation start\strut
\end{minipage}\tabularnewline
\begin{minipage}[t]{0.12\columnwidth}\raggedright\strut
\texttt{\%userinput}\strut
\end{minipage} & \begin{minipage}[t]{0.10\columnwidth}\raggedright\strut
Boolean is the current input from the user (vs the chatbot)\strut
\end{minipage}\tabularnewline
\begin{minipage}[t]{0.12\columnwidth}\raggedright\strut
\texttt{\%voice}\strut
\end{minipage} & \begin{minipage}[t]{0.10\columnwidth}\raggedright\strut
active or passive on current input\strut
\end{minipage}\tabularnewline
\bottomrule
\end{longtable}

\subsection{Chatbot Output}\label{chatbot-output}

\begin{longtable}[]{@{}ll@{}}
\toprule
\begin{minipage}[b]{0.12\columnwidth}\raggedright\strut
variable\strut
\end{minipage} & \begin{minipage}[b]{0.10\columnwidth}\raggedright\strut
description\strut
\end{minipage}\tabularnewline
\midrule
\endhead
\begin{minipage}[t]{0.12\columnwidth}\raggedright\strut
\texttt{\%inputrejoinder}\strut
\end{minipage} & \begin{minipage}[t]{0.10\columnwidth}\raggedright\strut
rule tag of any pending rejoinder for input or 0 if none\strut
\end{minipage}\tabularnewline
\begin{minipage}[t]{0.12\columnwidth}\raggedright\strut
\texttt{\%lastoutput}\strut
\end{minipage} & \begin{minipage}[t]{0.10\columnwidth}\raggedright\strut
the text of the last generated response for the current volley\strut
\end{minipage}\tabularnewline
\begin{minipage}[t]{0.12\columnwidth}\raggedright\strut
\texttt{\%lastquestion}\strut
\end{minipage} & \begin{minipage}[t]{0.10\columnwidth}\raggedright\strut
Boolean did last output end in a ?\strut
\end{minipage}\tabularnewline
\begin{minipage}[t]{0.12\columnwidth}\raggedright\strut
\texttt{\%outputrejoinder}\strut
\end{minipage} & \begin{minipage}[t]{0.10\columnwidth}\raggedright\strut
rule tag if system set a rejoinder for its current output or 0\strut
\end{minipage}\tabularnewline
\begin{minipage}[t]{0.12\columnwidth}\raggedright\strut
\texttt{\%response}\strut
\end{minipage} & \begin{minipage}[t]{0.10\columnwidth}\raggedright\strut
number of responses that have been generated for this sentence\strut
\end{minipage}\tabularnewline
\bottomrule
\end{longtable}

\subsection{System variables}\label{system-variables-1}

\begin{longtable}[]{@{}ll@{}}
\toprule
\begin{minipage}[b]{0.12\columnwidth}\raggedright\strut
variable\strut
\end{minipage} & \begin{minipage}[b]{0.10\columnwidth}\raggedright\strut
description\strut
\end{minipage}\tabularnewline
\midrule
\endhead
\begin{minipage}[t]{0.12\columnwidth}\raggedright\strut
\texttt{\%all}\strut
\end{minipage} & \begin{minipage}[t]{0.10\columnwidth}\raggedright\strut
Boolean is the :all flag on? (:all to set)\strut
\end{minipage}\tabularnewline
\begin{minipage}[t]{0.12\columnwidth}\raggedright\strut
\texttt{\%document}\strut
\end{minipage} & \begin{minipage}[t]{0.10\columnwidth}\raggedright\strut
Boolean is :document running\strut
\end{minipage}\tabularnewline
\begin{minipage}[t]{0.12\columnwidth}\raggedright\strut
\texttt{\%fact}\strut
\end{minipage} & \begin{minipage}[t]{0.10\columnwidth}\raggedright\strut
Numeric value most recent fact id\strut
\end{minipage}\tabularnewline
\begin{minipage}[t]{0.12\columnwidth}\raggedright\strut
\texttt{\%freetext}\strut
\end{minipage} & \begin{minipage}[t]{0.10\columnwidth}\raggedright\strut
kb of available text space\strut
\end{minipage}\tabularnewline
\begin{minipage}[t]{0.12\columnwidth}\raggedright\strut
\texttt{\%freedict}\strut
\end{minipage} & \begin{minipage}[t]{0.10\columnwidth}\raggedright\strut
number of unused dictionary words\strut
\end{minipage}\tabularnewline
\begin{minipage}[t]{0.12\columnwidth}\raggedright\strut
\texttt{\%freefact}\strut
\end{minipage} & \begin{minipage}[t]{0.10\columnwidth}\raggedright\strut
number of unused facts\strut
\end{minipage}\tabularnewline
\begin{minipage}[t]{0.12\columnwidth}\raggedright\strut
\texttt{\%maxmatchvariables}\strut
\end{minipage} & \begin{minipage}[t]{0.10\columnwidth}\raggedright\strut
highest number of \_match variables, currently 20\strut
\end{minipage}\tabularnewline
\begin{minipage}[t]{0.12\columnwidth}\raggedright\strut
\texttt{\%maxfactsets}\strut
\end{minipage} & \begin{minipage}[t]{0.10\columnwidth}\raggedright\strut
highest number of @factsets, currently 20\strut
\end{minipage}\tabularnewline
\begin{minipage}[t]{0.12\columnwidth}\raggedright\strut
\texttt{\%host}\strut
\end{minipage} & \begin{minipage}[t]{0.10\columnwidth}\raggedright\strut
name of the current host machine or ``local''\strut
\end{minipage}\tabularnewline
\begin{minipage}[t]{0.12\columnwidth}\raggedright\strut
\texttt{\%regression}\strut
\end{minipage} & \begin{minipage}[t]{0.10\columnwidth}\raggedright\strut
Boolean is the regression flag on\strut
\end{minipage}\tabularnewline
\begin{minipage}[t]{0.12\columnwidth}\raggedright\strut
\texttt{\%server}\strut
\end{minipage} & \begin{minipage}[t]{0.10\columnwidth}\raggedright\strut
Boolean is the system running in server mode\strut
\end{minipage}\tabularnewline
\begin{minipage}[t]{0.12\columnwidth}\raggedright\strut
\texttt{\%rule}\strut
\end{minipage} & \begin{minipage}[t]{0.10\columnwidth}\raggedright\strut
get a tag to the current executing rule. Can be used in place of a
label\strut
\end{minipage}\tabularnewline
\begin{minipage}[t]{0.12\columnwidth}\raggedright\strut
\texttt{\%topic}\strut
\end{minipage} & \begin{minipage}[t]{0.10\columnwidth}\raggedright\strut
name of the current ``real'' topic . if control is currently in a topic
or called from a topic which is not system or nostay, then that is the
topic. Otherwise the most recent pending topic is found\strut
\end{minipage}\tabularnewline
\begin{minipage}[t]{0.12\columnwidth}\raggedright\strut
\texttt{\%actualtopic}\strut
\end{minipage} & \begin{minipage}[t]{0.10\columnwidth}\raggedright\strut
literally the current topic being processed (system or not)\strut
\end{minipage}\tabularnewline
\begin{minipage}[t]{0.12\columnwidth}\raggedright\strut
\texttt{\%trace}\strut
\end{minipage} & \begin{minipage}[t]{0.10\columnwidth}\raggedright\strut
Numeric value of the trace flag (:trace to set)\strut
\end{minipage}\tabularnewline
\begin{minipage}[t]{0.12\columnwidth}\raggedright\strut
\texttt{\%httpresponse}\strut
\end{minipage} & \begin{minipage}[t]{0.10\columnwidth}\raggedright\strut
return code of most recent \^{}jsonopen call\strut
\end{minipage}\tabularnewline
\begin{minipage}[t]{0.12\columnwidth}\raggedright\strut
\texttt{\%pid}\strut
\end{minipage} & \begin{minipage}[t]{0.10\columnwidth}\raggedright\strut
Linux process id or 0 for other systems\strut
\end{minipage}\tabularnewline
\bottomrule
\end{longtable}

\subsection{Build data+}\label{build-data}

\begin{longtable}[]{@{}ll@{}}
\toprule
variable & description\tabularnewline
\midrule
\endhead
\texttt{\%dict} & date/time the dictionary was built\tabularnewline
\texttt{\%engine} & date/time the engine was compiled\tabularnewline
\texttt{\%os} & os invovled (linux windows mac ios)\tabularnewline
\texttt{\%script} & date/time build1 was compiled\tabularnewline
\texttt{\%version} & engine version number\tabularnewline
\bottomrule
\end{longtable}

You actually can assign to any of them. This will override them and make
them return what you tell them to and is a particularly BAD thing to do
if this is running on a server since it affects all users (unless you
reset the variable at the end of the volley. Assigning a period to a
variable resets it). Typically one does this as a temporary assignment
in a \#! comment line to set up conditions for testing using :verify.
Making them return a new value is NOT the same thing as making the
engine have a different value. Unless the variable is marked as
settable, setting a value affects only the value returned by a future
call to the system variable. It does not change engine values the
variable is meant to reflect.

\section{Control Over Input}\label{control-over-input}

The system can do a number of standard processing on user input,
including spell correction, proper-name merging, expanding contractions
etc. This is managed by setting the user variable \texttt{\$cs\_token}.

The default one that comes with Harry is:

\begin{verbatim}
$cs_token = #DO_INTERJECTION_SPLITTING | 
            #DO_SUBSTITUTE_SYSTEM |
            #DO_NUMBER_MERGE | 
            #DO_PROPERNAME_MERGE | 
            #DO_SPELLCHECK |
            #DO_PARSE
\end{verbatim}

The \texttt{\#}signals a named constant from the dictionarySystem.h
file. One can set the following:

These enable various LIVEDATA files to perform substitutions on input:

\begin{longtable}[]{@{}ll@{}}
\toprule
\begin{minipage}[b]{0.07\columnwidth}\raggedright\strut
flag\strut
\end{minipage} & \begin{minipage}[b]{0.10\columnwidth}\raggedright\strut
description\strut
\end{minipage}\tabularnewline
\midrule
\endhead
\begin{minipage}[t]{0.07\columnwidth}\raggedright\strut
\texttt{DO\_ESSENTIALS}\strut
\end{minipage} & \begin{minipage}[t]{0.10\columnwidth}\raggedright\strut
perform LIVEDATA/systemessentials which mostly strips off trailing
punctuation and sets corresponding flags instead\strut
\end{minipage}\tabularnewline
\begin{minipage}[t]{0.07\columnwidth}\raggedright\strut
\texttt{DO\_SUBSTITUTES}\strut
\end{minipage} & \begin{minipage}[t]{0.10\columnwidth}\raggedright\strut
perform LIVEDATA/substitutes\strut
\end{minipage}\tabularnewline
\begin{minipage}[t]{0.07\columnwidth}\raggedright\strut
\texttt{DO\_CONTRACTIONS}\strut
\end{minipage} & \begin{minipage}[t]{0.10\columnwidth}\raggedright\strut
perform LIVEDATA/contractions, expanding contractions\strut
\end{minipage}\tabularnewline
\begin{minipage}[t]{0.07\columnwidth}\raggedright\strut
\texttt{DO\_INTERJECTIONS}\strut
\end{minipage} & \begin{minipage}[t]{0.10\columnwidth}\raggedright\strut
perform LIVEDATA/interjections, changing phrases to interjections\strut
\end{minipage}\tabularnewline
\begin{minipage}[t]{0.07\columnwidth}\raggedright\strut
\texttt{DO\_BRITISH}\strut
\end{minipage} & \begin{minipage}[t]{0.10\columnwidth}\raggedright\strut
perform LIVEDATA/british, respelling brit words to American\strut
\end{minipage}\tabularnewline
\begin{minipage}[t]{0.07\columnwidth}\raggedright\strut
\texttt{DO\_SPELLING}\strut
\end{minipage} & \begin{minipage}[t]{0.10\columnwidth}\raggedright\strut
performs the LIVEDATA/spelling file (manual spell correction)\strut
\end{minipage}\tabularnewline
\begin{minipage}[t]{0.07\columnwidth}\raggedright\strut
\texttt{DO\_TEXTING}\strut
\end{minipage} & \begin{minipage}[t]{0.10\columnwidth}\raggedright\strut
performs the LIVEDATA/texting file (expand texting notation)\strut
\end{minipage}\tabularnewline
\begin{minipage}[t]{0.07\columnwidth}\raggedright\strut
\texttt{DO\_SUBSTITUTE\_SYSTEM}\strut
\end{minipage} & \begin{minipage}[t]{0.10\columnwidth}\raggedright\strut
do all LIVEDATA file expansions\strut
\end{minipage}\tabularnewline
\begin{minipage}[t]{0.07\columnwidth}\raggedright\strut
\texttt{DO\_INTERJECTION\_SPLITTING}\strut
\end{minipage} & \begin{minipage}[t]{0.10\columnwidth}\raggedright\strut
break off leading interjections into own sentence\strut
\end{minipage}\tabularnewline
\begin{minipage}[t]{0.07\columnwidth}\raggedright\strut
\texttt{DO\_NUMBER\_MERGE}\strut
\end{minipage} & \begin{minipage}[t]{0.10\columnwidth}\raggedright\strut
merge multiple word numbers into one (\emph{four and twenty})\strut
\end{minipage}\tabularnewline
\begin{minipage}[t]{0.07\columnwidth}\raggedright\strut
\texttt{DO\_PROPERNAME\_MERGE}\strut
\end{minipage} & \begin{minipage}[t]{0.10\columnwidth}\raggedright\strut
merge multiple proper name into one (\_George Harrison-)\strut
\end{minipage}\tabularnewline
\begin{minipage}[t]{0.07\columnwidth}\raggedright\strut
\texttt{DO\_DATE\_MERGE}\strut
\end{minipage} & \begin{minipage}[t]{0.10\columnwidth}\raggedright\strut
merge month day and/or year sequences (\emph{January 2, 1993})\strut
\end{minipage}\tabularnewline
\bottomrule
\end{longtable}

If any of the above items affect the input, they will be echoed as
values into \%tokenFlags so you can detect they happened. The next
changes do not echo into \%tokenFlags and relate to grammar of input:

\begin{longtable}[]{@{}ll@{}}
\toprule
\begin{minipage}[b]{0.07\columnwidth}\raggedright\strut
flag\strut
\end{minipage} & \begin{minipage}[b]{0.10\columnwidth}\raggedright\strut
description\strut
\end{minipage}\tabularnewline
\midrule
\endhead
\begin{minipage}[t]{0.07\columnwidth}\raggedright\strut
\texttt{DO\_POSTAG}\strut
\end{minipage} & \begin{minipage}[t]{0.10\columnwidth}\raggedright\strut
allow pos-tagging (labels like \textasciitilde{}noun
\textasciitilde{}verb become marked)\strut
\end{minipage}\tabularnewline
\begin{minipage}[t]{0.07\columnwidth}\raggedright\strut
\texttt{DO\_PARSE}\strut
\end{minipage} & \begin{minipage}[t]{0.10\columnwidth}\raggedright\strut
allow parser (labels for word roles like
\textasciitilde{}main\_subject)\strut
\end{minipage}\tabularnewline
\begin{minipage}[t]{0.07\columnwidth}\raggedright\strut
\texttt{DO\_CONDITIONAL\_POSTAG}\strut
\end{minipage} & \begin{minipage}[t]{0.10\columnwidth}\raggedright\strut
perform pos-tagging only if all words are known. Avoids wasting time on
foreign sentences in particular\strut
\end{minipage}\tabularnewline
\begin{minipage}[t]{0.07\columnwidth}\raggedright\strut
\texttt{NO\_ERASE}\strut
\end{minipage} & \begin{minipage}[t]{0.10\columnwidth}\raggedright\strut
where a substitution would delete a word entirely as junk, don't\strut
\end{minipage}\tabularnewline
\begin{minipage}[t]{0.07\columnwidth}\raggedright\strut
\texttt{DO\_SPLIT\_UNDERSCORES}\strut
\end{minipage} & \begin{minipage}[t]{0.10\columnwidth}\raggedright\strut
happens after all other input tokenization and adjustments except number
merge, and separates words that have been conjoined either because the
dictionary has them (\emph{credit\_card}) or because they were merged by
proper name merging, or by substitution. The result is only words
without underscores (excluding number words like
\emph{five\_thousand\_and\_four}\strut
\end{minipage}\tabularnewline
\begin{minipage}[t]{0.07\columnwidth}\raggedright\strut
\texttt{MARK\_LOWER}\strut
\end{minipage} & \begin{minipage}[t]{0.10\columnwidth}\raggedright\strut
if a word is considered a proper name in CS and is marked as an upper
case word, this will force it to perform any markings for its lower case
form as well. Sometimes users type stuff in upper case that really
should be lower\strut
\end{minipage}\tabularnewline
\bottomrule
\end{longtable}

Normally the system tries to outguess the user, who cannot be trusted to
use correct punctuation or casing or spelling. These block that:

\begin{longtable}[]{@{}ll@{}}
\toprule
\begin{minipage}[b]{0.07\columnwidth}\raggedright\strut
flag\strut
\end{minipage} & \begin{minipage}[b]{0.10\columnwidth}\raggedright\strut
description\strut
\end{minipage}\tabularnewline
\midrule
\endhead
\begin{minipage}[t]{0.07\columnwidth}\raggedright\strut
\texttt{STRICT\_CASING}\strut
\end{minipage} & \begin{minipage}[t]{0.10\columnwidth}\raggedright\strut
except for 1st word of a sentence, assume user uses correct casing on
words\strut
\end{minipage}\tabularnewline
\begin{minipage}[t]{0.07\columnwidth}\raggedright\strut
\texttt{NO\_INFER\_QUESTION}\strut
\end{minipage} & \begin{minipage}[t]{0.10\columnwidth}\raggedright\strut
the system will not try to set the QUESTIONMARK flag if the user didn't
input a ? and the structure of the input looks like a question\strut
\end{minipage}\tabularnewline
\begin{minipage}[t]{0.07\columnwidth}\raggedright\strut
\texttt{DO\_SPELLCHECK}\strut
\end{minipage} & \begin{minipage}[t]{0.10\columnwidth}\raggedright\strut
perform internal spell checking\strut
\end{minipage}\tabularnewline
\begin{minipage}[t]{0.07\columnwidth}\raggedright\strut
\texttt{ONLY\_LOWERCASE}\strut
\end{minipage} & \begin{minipage}[t]{0.10\columnwidth}\raggedright\strut
force all input (except ``I'') to be lower case, refuse to recognize
uppercase forms of anything\strut
\end{minipage}\tabularnewline
\begin{minipage}[t]{0.07\columnwidth}\raggedright\strut
\texttt{NO\_IMPERATIVE}\strut
\end{minipage} & \begin{minipage}[t]{0.10\columnwidth}\raggedright\strut
\strut
\end{minipage}\tabularnewline
\begin{minipage}[t]{0.07\columnwidth}\raggedright\strut
\texttt{NO\_WITHIN}\strut
\end{minipage} & \begin{minipage}[t]{0.10\columnwidth}\raggedright\strut
\strut
\end{minipage}\tabularnewline
\begin{minipage}[t]{0.07\columnwidth}\raggedright\strut
\texttt{NO\_SENTENCE\_END}\strut
\end{minipage} & \begin{minipage}[t]{0.10\columnwidth}\raggedright\strut
\strut
\end{minipage}\tabularnewline
\bottomrule
\end{longtable}

Normally the tokenizer breaks apart some kinds of sentences into two.
These prevent that:

\begin{longtable}[]{@{}ll@{}}
\toprule
\begin{minipage}[b]{0.07\columnwidth}\raggedright\strut
flag\strut
\end{minipage} & \begin{minipage}[b]{0.10\columnwidth}\raggedright\strut
description\strut
\end{minipage}\tabularnewline
\midrule
\endhead
\begin{minipage}[t]{0.07\columnwidth}\raggedright\strut
\texttt{NO\_COLON\_END}\strut
\end{minipage} & \begin{minipage}[t]{0.10\columnwidth}\raggedright\strut
don't break apart a sentence after a colon\strut
\end{minipage}\tabularnewline
\begin{minipage}[t]{0.07\columnwidth}\raggedright\strut
\texttt{NO\_SEMICOLON\_END}\strut
\end{minipage} & \begin{minipage}[t]{0.10\columnwidth}\raggedright\strut
don't break apart a sentence after a semi-colon\strut
\end{minipage}\tabularnewline
\begin{minipage}[t]{0.07\columnwidth}\raggedright\strut
\texttt{UNTOUCHED\_INPUT}\strut
\end{minipage} & \begin{minipage}[t]{0.10\columnwidth}\raggedright\strut
if set to this alone, will tokenize only on spaces, leaving everything
but spacing untouched\strut
\end{minipage}\tabularnewline
\begin{minipage}[t]{0.07\columnwidth}\raggedright\strut
\texttt{LEAVE\_QUOTE}\strut
\end{minipage} & \begin{minipage}[t]{0.10\columnwidth}\raggedright\strut
if input is found withing " " it will become a single token exactly as
it is seen. W/o Leave\_Quote, it is converted into a word without quotes
and using underscores instead of spaces. So ``My Fair Lady'' becomes
My\_Fair\_Lady, which would match a movie title if you had one, unlike
\emph{My Fair Lady} becoming the resulting token and unrecognized\strut
\end{minipage}\tabularnewline
\bottomrule
\end{longtable}

Note, you can change \texttt{\$cs\_token} on the fly and force input to
be reanalyzed via \texttt{\^{}retry(SENTENCE)}. I do this when I detect
the user is trying to give his name, and many foreign names might be
spell-corrected into something wrong and the user is unlikely to
misspell his own name. Just remember to reset \texttt{\$cs\_token} back
to normal after you are done. Here is one such way, assuming
\texttt{\$stdtoken} is set to your normal tokenflags in your bot
definition outputmacro:

\begin{verbatim}
#! my name is Rogr
s: (name is _*)

    if ($cs_token == $stdtoken)
        {
        $cs_token = #DO_INTERJECTION_SPLITTING |
                    #DO_SUBSTITUTE_SYSTEM | #DO_NUMBER_MERGE |
                    #DO_PARSE
        retry(SENTENCE)
        }
    _0 is the name.
    $cs_token = $stdtoken
\end{verbatim}

If you type \emph{my name is Rogr} into a topic with this, the original
input is spell-corrected to \emph{my name is Roger}, but this will
change the \texttt{\$cs\_token} over to one without spell correction and
redo the sentence, which will now come back with \emph{my name is Rogr}
and be echoed correctly, and \texttt{\$cs\_token\ reset}. That's
assuming nothing else would run differently and trap the response
elsewhere. If you were worried about that, it would be possible for the
script to save where it is using \texttt{\^{}getrule(tag)} and modify
your control script to return immediate control to here after input
processing if you had changed \texttt{\$cs\_token}.

\subsection{Private Substitutions}\label{private-substitutions}

While in general, substitutions are defined in the LIVEDATA folder, you
can define private substititions for your specific bot using the
scripting language. You can say

\begin{verbatim}
replace: xxx yyyyy
\end{verbatim}

which defines a substitution just like a livedata substitution file. It
actually creates a substitution file called \texttt{private0.txt} or
\texttt{private1.txt} in your TOPIC folder. Even then, those
substitutions will not be enacted unless you explicitly add to the
\$cs\_token value \#DO\_PRIVATE, eg

\begin{verbatim}
$cs_token = #DO_INTERJECTION_SPLITTING | 
            #DO_SUBSTITUTE_SYSTEM |
            #DO_NUMBER_MERGE | 
            #DO_PROPERNAME_MERGE |
            #DO_SPELLCHECK | 
            #DO_PARSE | 
            #DO_PRIVATE
\end{verbatim}

Similarly while canonical values of words can be defined in
\texttt{LIVEDATA/SYSTEM/canonical.txt}, you can define private canonical
values for your bots by using the scripting language. You can say:

\begin{verbatim}
canon: oh 0 faster fast
\end{verbatim}

which defines new canonical values for things and creates a file
canon0.txt or canon1.txt in your TOPIC folder. If you want to set a
canonical pair from a table during compilation, you can use a function
to do the same thing (but only 1 pair at a time).

\begin{verbatim}
^canon(word canonicalform)
\end{verbatim}

\section{Interchange Variables}\label{interchange-variables}

The following variables can be defined in a script and the engine will
react to their contents.

\begin{longtable}[]{@{}ll@{}}
\toprule
\begin{minipage}[b]{0.26\columnwidth}\raggedright\strut
interchange variable\strut
\end{minipage} & \begin{minipage}[b]{0.10\columnwidth}\raggedright\strut
description\strut
\end{minipage}\tabularnewline
\midrule
\endhead
\begin{minipage}[t]{0.26\columnwidth}\raggedright\strut
\texttt{\$cs\_token}\strut
\end{minipage} & \begin{minipage}[t]{0.10\columnwidth}\raggedright\strut
described extensively above\strut
\end{minipage}\tabularnewline
\begin{minipage}[t]{0.26\columnwidth}\raggedright\strut
\texttt{\$cs\_response}\strut
\end{minipage} & \begin{minipage}[t]{0.10\columnwidth}\raggedright\strut
controls automatic handling of outputs to user. By default it consists
of
\texttt{\$cs\_response\ =\ \#Response\_upperstart\ \textbar{}\ \#response\_removespacebeforecomma\ \textbar{}\ \#response\_alterunderscores\ \textbar{}\ \#response\_removetilde}
If you want none of theses, use \$cs\_response = 0 (all flags turned
off). See \^{}print for explanation of flags.
\texttt{\#response\_upperstart} -- makes the first letter of an output
sentence capitalized, \texttt{\#Response\_removespacebeforecomma} --
does the obvious, \texttt{\#Response\_alterunderscores} - converts
single underscores to spaces and double underscores to singles (eg for a
web url)\strut
\end{minipage}\tabularnewline
\begin{minipage}[t]{0.26\columnwidth}\raggedright\strut
\texttt{\$cs\_jsontimeout}\strut
\end{minipage} & \begin{minipage}[t]{0.10\columnwidth}\raggedright\strut
seconds before JsonOpen declares a time out failure. If unspecified the
default is 300\strut
\end{minipage}\tabularnewline
\begin{minipage}[t]{0.26\columnwidth}\raggedright\strut
\texttt{\$cs\_crashmsg}\strut
\end{minipage} & \begin{minipage}[t]{0.10\columnwidth}\raggedright\strut
in server mode, what to say if the server crashes and we return a
message to the user. By default the message is \emph{Hey, sorry. I
forgot what I was thinking about.}\strut
\end{minipage}\tabularnewline
\begin{minipage}[t]{0.26\columnwidth}\raggedright\strut
\texttt{\$cs\_abstract}\strut
\end{minipage} & \begin{minipage}[t]{0.10\columnwidth}\raggedright\strut
used with :abstract\strut
\end{minipage}\tabularnewline
\begin{minipage}[t]{0.26\columnwidth}\raggedright\strut
\texttt{\$cs\_looplimit}\strut
\end{minipage} & \begin{minipage}[t]{0.10\columnwidth}\raggedright\strut
loop() defaults to 1000 iterations before stopping. You can change this
default with this\strut
\end{minipage}\tabularnewline
\begin{minipage}[t]{0.26\columnwidth}\raggedright\strut
\texttt{\$cs\_trace}\strut
\end{minipage} & \begin{minipage}[t]{0.10\columnwidth}\raggedright\strut
if this variable is defined, then whenever the user's volley is
finished, the value of this variable is set to that of :trace and :trace
is cleared to 0, but when the user is read back in, the :trace is set to
this value. For a server, this means you can perform tracing on a user
w/o making all user transactions dump trace data\strut
\end{minipage}\tabularnewline
\begin{minipage}[t]{0.26\columnwidth}\raggedright\strut
\texttt{\$cs\_control\_pre}\strut
\end{minipage} & \begin{minipage}[t]{0.10\columnwidth}\raggedright\strut
name of topic to run in gambit mode on pre-pass, set by author. Runs
before any sentences of the input volley are analyzed. Good for setting
up initial values\strut
\end{minipage}\tabularnewline
\begin{minipage}[t]{0.26\columnwidth}\raggedright\strut
\texttt{\$cs\_usermessagelimit}\strut
\end{minipage} & \begin{minipage}[t]{0.10\columnwidth}\raggedright\strut
max number of message pairs (user input \& bot output) saved in topic
file\strut
\end{minipage}\tabularnewline
\begin{minipage}[t]{0.26\columnwidth}\raggedright\strut
\texttt{\$cs\_externaltag}\strut
\end{minipage} & \begin{minipage}[t]{0.10\columnwidth}\raggedright\strut
name of a topic to use to replace existing internal English pos-parser.
See bottom of ChatScript PosParser manual for details\strut
\end{minipage}\tabularnewline
\begin{minipage}[t]{0.26\columnwidth}\raggedright\strut
\texttt{\$cs\_prepass}\strut
\end{minipage} & \begin{minipage}[t]{0.10\columnwidth}\raggedright\strut
name of a topic to run in responder mode on main volleys, which runs
before \$cs\_control\_main and after all of the above and pos-parsing is
done. Used to amend preparation data coming from the engine. You can use
it to add your own spin on input processing before going to your main
control. I use it to, for example, label commands as questions,
standardize sentence construction (like \emph{if you see me what will
you think} to \emph{assume you see me. What will you think?})\strut
\end{minipage}\tabularnewline
\begin{minipage}[t]{0.26\columnwidth}\raggedright\strut
\texttt{\$cs\_control\_main}\strut
\end{minipage} & \begin{minipage}[t]{0.10\columnwidth}\raggedright\strut
name of topic to run in responder mode on main volleys, set by
author\strut
\end{minipage}\tabularnewline
\begin{minipage}[t]{0.26\columnwidth}\raggedright\strut
\texttt{\$cs\_control\_post}\strut
\end{minipage} & \begin{minipage}[t]{0.10\columnwidth}\raggedright\strut
name of topic to run in gambit mode on post-pass, set by author\strut
\end{minipage}\tabularnewline
\begin{minipage}[t]{0.26\columnwidth}\raggedright\strut
\texttt{\$botprompt}\strut
\end{minipage} & \begin{minipage}[t]{0.10\columnwidth}\raggedright\strut
message for console window to label bot output\strut
\end{minipage}\tabularnewline
\begin{minipage}[t]{0.26\columnwidth}\raggedright\strut
\texttt{\$userprompt}\strut
\end{minipage} & \begin{minipage}[t]{0.10\columnwidth}\raggedright\strut
message for console window to label user input line\strut
\end{minipage}\tabularnewline
\begin{minipage}[t]{0.26\columnwidth}\raggedright\strut
\texttt{\$cs\_crashmsg}\strut
\end{minipage} & \begin{minipage}[t]{0.10\columnwidth}\raggedright\strut
message to use if a server crash occurs\strut
\end{minipage}\tabularnewline
\begin{minipage}[t]{0.26\columnwidth}\raggedright\strut
\texttt{\$cs\_language}\strut
\end{minipage} & \begin{minipage}[t]{0.10\columnwidth}\raggedright\strut
if spanish, will adjust spell checking for spanish colloquial\strut
\end{minipage}\tabularnewline
\begin{minipage}[t]{0.26\columnwidth}\raggedright\strut
\texttt{\$cs\_token}\strut
\end{minipage} & \begin{minipage}[t]{0.10\columnwidth}\raggedright\strut
bits controlling how the tokenizer works. By default when null, you get
all bits assumed on. The possible values are in src/dictionarySystem.h
(hunt for \$token) and you put a \# in front of them to generate that
named numeric constant\strut
\end{minipage}\tabularnewline
\begin{minipage}[t]{0.26\columnwidth}\raggedright\strut
\texttt{\$cs\_abstract}\strut
\end{minipage} & \begin{minipage}[t]{0.10\columnwidth}\raggedright\strut
topic used by :abstract to display facts if you want them
displayed\strut
\end{minipage}\tabularnewline
\begin{minipage}[t]{0.26\columnwidth}\raggedright\strut
\texttt{\$cs\_prepass}\strut
\end{minipage} & \begin{minipage}[t]{0.10\columnwidth}\raggedright\strut
topic used between parsing and running user control script. Useful to
supplement parsing, setting the question value, and revising input
idioms\strut
\end{minipage}\tabularnewline
\begin{minipage}[t]{0.26\columnwidth}\raggedright\strut
\texttt{\$cs\_wildcardseparator}\strut
\end{minipage} & \begin{minipage}[t]{0.10\columnwidth}\raggedright\strut
when a match variable covers multiple words, what should separate them-
by default it's a space, but underscore is handy too. Initial system
character is space, creating fidelity with what was typed. Useful if \_
can be recognized in input (web addresses). Changing to \_ is consistent
with multi-word representation and keyword recognition for concepts. CS
automatically converts \_ to space on output, so internal use of \_ is
normal\strut
\end{minipage}\tabularnewline
\begin{minipage}[t]{0.26\columnwidth}\raggedright\strut
\texttt{\$cs\_userfactlimit}\strut
\end{minipage} & \begin{minipage}[t]{0.10\columnwidth}\raggedright\strut
how many of the most recent permanent facts created by the script in
response to user inputs are kept for each user. Std default is 100\strut
\end{minipage}\tabularnewline
\begin{minipage}[t]{0.26\columnwidth}\raggedright\strut
\texttt{\$cs\_response}\strut
\end{minipage} & \begin{minipage}[t]{0.10\columnwidth}\raggedright\strut
controls some characteristics of how responses are formatted\strut
\end{minipage}\tabularnewline
\begin{minipage}[t]{0.26\columnwidth}\raggedright\strut
\texttt{\$cs\_randIndex}\strut
\end{minipage} & \begin{minipage}[t]{0.10\columnwidth}\raggedright\strut
the random seed for this volley\strut
\end{minipage}\tabularnewline
\begin{minipage}[t]{0.26\columnwidth}\raggedright\strut
\texttt{\$cs\_utcoffset}\strut
\end{minipage} & \begin{minipage}[t]{0.10\columnwidth}\raggedright\strut
if defined, then \%time returns current utc time + timezone offset. The
offset is usually a simple number, meaning hours, and can have + or --
in front of it. It can also be a normal time reference like 02:30 which
means plus 2 hours and 30 minutes beyond utc, or -01:30:20 which means 1
hour, 30 minutes, and 20 seconds before utc (as if anyone would use
that). The following variables are generated by the system on behalf of
scripts\strut
\end{minipage}\tabularnewline
\begin{minipage}[t]{0.26\columnwidth}\raggedright\strut
\texttt{\$\$db\_error}\strut
\end{minipage} & \begin{minipage}[t]{0.10\columnwidth}\raggedright\strut
error message from a postgres failure \$\$findtext\_start - \^{}findtext
return the end normally, this is where it puts the start\strut
\end{minipage}\tabularnewline
\begin{minipage}[t]{0.26\columnwidth}\raggedright\strut
\texttt{\$\$tcpopen\_error}\strut
\end{minipage} & \begin{minipage}[t]{0.10\columnwidth}\raggedright\strut
error message from a tcpopen error\strut
\end{minipage}\tabularnewline
\begin{minipage}[t]{0.26\columnwidth}\raggedright\strut
\texttt{\$\$document}\strut
\end{minipage} & \begin{minipage}[t]{0.10\columnwidth}\raggedright\strut
name of the document being read in document mode\strut
\end{minipage}\tabularnewline
\begin{minipage}[t]{0.26\columnwidth}\raggedright\strut
\texttt{\$cs\_randindex}\strut
\end{minipage} & \begin{minipage}[t]{0.10\columnwidth}\raggedright\strut
current value of the random generator value\strut
\end{minipage}\tabularnewline
\begin{minipage}[t]{0.26\columnwidth}\raggedright\strut
\texttt{\$cs\_bot}\strut
\end{minipage} & \begin{minipage}[t]{0.10\columnwidth}\raggedright\strut
name of the bot currently in use\strut
\end{minipage}\tabularnewline
\begin{minipage}[t]{0.26\columnwidth}\raggedright\strut
\texttt{\$cs\_login}\strut
\end{minipage} & \begin{minipage}[t]{0.10\columnwidth}\raggedright\strut
login name of the user\strut
\end{minipage}\tabularnewline
\bottomrule
\end{longtable}

\end{document}
